% Options for packages loaded elsewhere
\PassOptionsToPackage{unicode}{hyperref}
\PassOptionsToPackage{hyphens}{url}
%
\documentclass[
]{article}
\usepackage{lmodern}
\usepackage{amssymb,amsmath}
\usepackage{ifxetex,ifluatex}
\ifnum 0\ifxetex 1\fi\ifluatex 1\fi=0 % if pdftex
  \usepackage[T1]{fontenc}
  \usepackage[utf8]{inputenc}
  \usepackage{textcomp} % provide euro and other symbols
\else % if luatex or xetex
  \usepackage{unicode-math}
  \defaultfontfeatures{Scale=MatchLowercase}
  \defaultfontfeatures[\rmfamily]{Ligatures=TeX,Scale=1}
\fi
% Use upquote if available, for straight quotes in verbatim environments
\IfFileExists{upquote.sty}{\usepackage{upquote}}{}
\IfFileExists{microtype.sty}{% use microtype if available
  \usepackage[]{microtype}
  \UseMicrotypeSet[protrusion]{basicmath} % disable protrusion for tt fonts
}{}
\makeatletter
\@ifundefined{KOMAClassName}{% if non-KOMA class
  \IfFileExists{parskip.sty}{%
    \usepackage{parskip}
  }{% else
    \setlength{\parindent}{0pt}
    \setlength{\parskip}{6pt plus 2pt minus 1pt}}
}{% if KOMA class
  \KOMAoptions{parskip=half}}
\makeatother
\usepackage{xcolor}
\IfFileExists{xurl.sty}{\usepackage{xurl}}{} % add URL line breaks if available
\IfFileExists{bookmark.sty}{\usepackage{bookmark}}{\usepackage{hyperref}}
\hypersetup{
  pdftitle={Demonstration of netCDF functionality in rSW2analysis},
  pdfauthor={Caitlin M. Andrews},
  hidelinks,
  pdfcreator={LaTeX via pandoc}}
\urlstyle{same} % disable monospaced font for URLs
\usepackage[margin=1in]{geometry}
\usepackage{color}
\usepackage{fancyvrb}
\newcommand{\VerbBar}{|}
\newcommand{\VERB}{\Verb[commandchars=\\\{\}]}
\DefineVerbatimEnvironment{Highlighting}{Verbatim}{commandchars=\\\{\}}
% Add ',fontsize=\small' for more characters per line
\usepackage{framed}
\definecolor{shadecolor}{RGB}{248,248,248}
\newenvironment{Shaded}{\begin{snugshade}}{\end{snugshade}}
\newcommand{\AlertTok}[1]{\textcolor[rgb]{0.94,0.16,0.16}{#1}}
\newcommand{\AnnotationTok}[1]{\textcolor[rgb]{0.56,0.35,0.01}{\textbf{\textit{#1}}}}
\newcommand{\AttributeTok}[1]{\textcolor[rgb]{0.77,0.63,0.00}{#1}}
\newcommand{\BaseNTok}[1]{\textcolor[rgb]{0.00,0.00,0.81}{#1}}
\newcommand{\BuiltInTok}[1]{#1}
\newcommand{\CharTok}[1]{\textcolor[rgb]{0.31,0.60,0.02}{#1}}
\newcommand{\CommentTok}[1]{\textcolor[rgb]{0.56,0.35,0.01}{\textit{#1}}}
\newcommand{\CommentVarTok}[1]{\textcolor[rgb]{0.56,0.35,0.01}{\textbf{\textit{#1}}}}
\newcommand{\ConstantTok}[1]{\textcolor[rgb]{0.00,0.00,0.00}{#1}}
\newcommand{\ControlFlowTok}[1]{\textcolor[rgb]{0.13,0.29,0.53}{\textbf{#1}}}
\newcommand{\DataTypeTok}[1]{\textcolor[rgb]{0.13,0.29,0.53}{#1}}
\newcommand{\DecValTok}[1]{\textcolor[rgb]{0.00,0.00,0.81}{#1}}
\newcommand{\DocumentationTok}[1]{\textcolor[rgb]{0.56,0.35,0.01}{\textbf{\textit{#1}}}}
\newcommand{\ErrorTok}[1]{\textcolor[rgb]{0.64,0.00,0.00}{\textbf{#1}}}
\newcommand{\ExtensionTok}[1]{#1}
\newcommand{\FloatTok}[1]{\textcolor[rgb]{0.00,0.00,0.81}{#1}}
\newcommand{\FunctionTok}[1]{\textcolor[rgb]{0.00,0.00,0.00}{#1}}
\newcommand{\ImportTok}[1]{#1}
\newcommand{\InformationTok}[1]{\textcolor[rgb]{0.56,0.35,0.01}{\textbf{\textit{#1}}}}
\newcommand{\KeywordTok}[1]{\textcolor[rgb]{0.13,0.29,0.53}{\textbf{#1}}}
\newcommand{\NormalTok}[1]{#1}
\newcommand{\OperatorTok}[1]{\textcolor[rgb]{0.81,0.36,0.00}{\textbf{#1}}}
\newcommand{\OtherTok}[1]{\textcolor[rgb]{0.56,0.35,0.01}{#1}}
\newcommand{\PreprocessorTok}[1]{\textcolor[rgb]{0.56,0.35,0.01}{\textit{#1}}}
\newcommand{\RegionMarkerTok}[1]{#1}
\newcommand{\SpecialCharTok}[1]{\textcolor[rgb]{0.00,0.00,0.00}{#1}}
\newcommand{\SpecialStringTok}[1]{\textcolor[rgb]{0.31,0.60,0.02}{#1}}
\newcommand{\StringTok}[1]{\textcolor[rgb]{0.31,0.60,0.02}{#1}}
\newcommand{\VariableTok}[1]{\textcolor[rgb]{0.00,0.00,0.00}{#1}}
\newcommand{\VerbatimStringTok}[1]{\textcolor[rgb]{0.31,0.60,0.02}{#1}}
\newcommand{\WarningTok}[1]{\textcolor[rgb]{0.56,0.35,0.01}{\textbf{\textit{#1}}}}
\usepackage{longtable,booktabs}
% Correct order of tables after \paragraph or \subparagraph
\usepackage{etoolbox}
\makeatletter
\patchcmd\longtable{\par}{\if@noskipsec\mbox{}\fi\par}{}{}
\makeatother
% Allow footnotes in longtable head/foot
\IfFileExists{footnotehyper.sty}{\usepackage{footnotehyper}}{\usepackage{footnote}}
\makesavenoteenv{longtable}
\usepackage{graphicx}
\makeatletter
\def\maxwidth{\ifdim\Gin@nat@width>\linewidth\linewidth\else\Gin@nat@width\fi}
\def\maxheight{\ifdim\Gin@nat@height>\textheight\textheight\else\Gin@nat@height\fi}
\makeatother
% Scale images if necessary, so that they will not overflow the page
% margins by default, and it is still possible to overwrite the defaults
% using explicit options in \includegraphics[width, height, ...]{}
\setkeys{Gin}{width=\maxwidth,height=\maxheight,keepaspectratio}
% Set default figure placement to htbp
\makeatletter
\def\fps@figure{htbp}
\makeatother
\setlength{\emergencystretch}{3em} % prevent overfull lines
\providecommand{\tightlist}{%
  \setlength{\itemsep}{0pt}\setlength{\parskip}{0pt}}
\setcounter{secnumdepth}{5}
\ifluatex
  \usepackage{selnolig}  % disable illegal ligatures
\fi

\title{Demonstration of netCDF functionality in rSW2analysis}
\author{Caitlin M. Andrews}
\date{2020-08-26}

\begin{document}
\maketitle

{
\setcounter{tocdepth}{2}
\tableofcontents
}
\hypertarget{overview}{%
\section{Overview}\label{overview}}

This vignette is an overview of how to use the functions in the rSW2analysis package to
convert arrays of rSOILWAT2 and SOILWAT2 output data into netCDFs that adhere to CF conventions.

\hypertarget{input-data-format-and-structure}{%
\section{Input data format and structure}\label{input-data-format-and-structure}}

\hypertarget{variable-data}{%
\subsection{Variable data}\label{variable-data}}

The netCDF functions in this package expect an array, in which each row represents
data for a site. Columns represent one of three things:
* multiple different variables for one site
* one variable with multiple measurements (time series)
* one variable with multiple depth measurements.

\begin{Shaded}
\begin{Highlighting}[]
   \CommentTok{\# dummy data for 5 sites, 10 data points (could be 10 vars, 10 times, or 10 depths)}
\NormalTok{   someData \textless{}{-}}\StringTok{ }\KeywordTok{rnorm}\NormalTok{(}\DecValTok{50}\NormalTok{, }\DecValTok{7}\NormalTok{, }\DecValTok{30}\NormalTok{)}
\NormalTok{   data1d \textless{}{-}}\StringTok{ }\KeywordTok{array}\NormalTok{(someData, }\KeywordTok{c}\NormalTok{(}\DecValTok{5}\NormalTok{, }\DecValTok{10}\NormalTok{))}
   \KeywordTok{str}\NormalTok{(data1d)}
   \KeywordTok{is.array}\NormalTok{(data1d)}
\end{Highlighting}
\end{Shaded}

Additionally, it is possible to have data at multiple times \emph{and} multiple depths.
In this case, time should be organized by column, and multiple depths should be organized
sequentially in third dimension, forming a 3d array.

\begin{Shaded}
\begin{Highlighting}[]
  \CommentTok{\# dummy data for 5 sites, 10 times, 3 depths.}
\NormalTok{  data3d \textless{}{-}}\StringTok{ }\KeywordTok{array}\NormalTok{(}\DecValTok{0}\NormalTok{, }\KeywordTok{c}\NormalTok{(}\DecValTok{5}\NormalTok{, }\DecValTok{10}\NormalTok{, }\DecValTok{3}\NormalTok{))}
\NormalTok{  data3d[,,}\DecValTok{1}\NormalTok{] \textless{}{-}}\StringTok{ }\NormalTok{someData}
\NormalTok{  data3d[,,}\DecValTok{2}\NormalTok{] \textless{}{-}}\StringTok{ }\NormalTok{someData}
\NormalTok{  data3d[,,}\DecValTok{3}\NormalTok{] \textless{}{-}}\StringTok{ }\NormalTok{someData}
  \KeywordTok{str}\NormalTok{(data3d)}
  \KeywordTok{is.array}\NormalTok{(data3d)}
\end{Highlighting}
\end{Shaded}

\hypertarget{location-data}{%
\subsection{Location data}\label{location-data}}

Locations all need to be on the same coordinates reference system (CRS). Data can
be either be organized on a regularly spaced grid (i.e.~wall to wall simulations for
a given area) or not (i.e.~project data representing plot points in a park).

If the former, the coordinates need to be in order. If the later,

\begin{Shaded}
\begin{Highlighting}[]
  \CommentTok{\# points on a regular spaced grid in the WGS84 CRS}
\NormalTok{  locationsGrid \textless{}{-}}\StringTok{ }\KeywordTok{data.frame}\NormalTok{(}\DataTypeTok{X\_WGS84 =} \KeywordTok{c}\NormalTok{(}\KeywordTok{rep}\NormalTok{(}\OperatorTok{{-}}\FloatTok{124.5938}\NormalTok{, }\DecValTok{5}\NormalTok{)),}
                        \DataTypeTok{Y\_WGS84 =} \KeywordTok{c}\NormalTok{(}\FloatTok{47.90625}\NormalTok{, }\FloatTok{47.96875}\NormalTok{, }\FloatTok{48.03125}\NormalTok{, }\FloatTok{48.09375}\NormalTok{,}
                                     \FloatTok{48.15625}\NormalTok{))}
   
  \CommentTok{\# irregularly spaced points in the WGS84 CRS}
  \CommentTok{\#locationsRandom}
\end{Highlighting}
\end{Shaded}

\hypertarget{attribute-data}{%
\subsection{Attribute data}\label{attribute-data}}

Attribute data pertains to the metadata contained in the netCDFs. NetCDFs are
self-documenting and require certain metadata attributes, providing information
necessary for interpreting the data. Different attribute types include variable attributes,
time attributes, depth attributes, and global attributes.

Full documentation of the different attributes, and whether they are required or
recommended can be found in the Dryland Ecology's netCDF Conventions and Standards
documentation.

\hypertarget{variable-attributes}{%
\subsubsection{Variable Attributes}\label{variable-attributes}}

A named list describing the variable data.

\begin{Shaded}
\begin{Highlighting}[]
\NormalTok{var\_attributes \textless{}{-}}\StringTok{ }\KeywordTok{list}\NormalTok{(}
  \DataTypeTok{name =} \StringTok{\textquotesingle{}WatYrWDD\textquotesingle{}}\NormalTok{,}
  \DataTypeTok{long\_name =} \StringTok{\textquotesingle{}Water Year Wet Degree Days\textquotesingle{}}\NormalTok{,}
  \DataTypeTok{units =} \StringTok{\textquotesingle{}number of days\textquotesingle{}}
\NormalTok{)}
\end{Highlighting}
\end{Shaded}

\hypertarget{time-attributes}{%
\subsubsection{Time Attributes}\label{time-attributes}}

A named list describing the time dimension.

A common unit for netCDF is days since 1900. A full list of accepted time units can
be found \href{https://ncics.org/portfolio/other-resources/udunits2/}{here}.

\begin{Shaded}
\begin{Highlighting}[]
\NormalTok{time\_attributes \textless{}{-}}\StringTok{ }\KeywordTok{list}\NormalTok{(}
  \DataTypeTok{name =} \StringTok{\textquotesingle{}time\textquotesingle{}}\NormalTok{,}
  \DataTypeTok{units =} \StringTok{\textquotesingle{}days since 1900{-}01{-}01\textquotesingle{}}\NormalTok{,}
  \DataTypeTok{calendar =} \StringTok{\textquotesingle{}standard\textquotesingle{}}\NormalTok{,}
  \DataTypeTok{unlim =} \OtherTok{TRUE}\NormalTok{,}
  \DataTypeTok{vals =} \KeywordTok{c}\NormalTok{(}\DecValTok{43554}\NormalTok{, }\DecValTok{43920}\NormalTok{, }\DecValTok{44285}\NormalTok{, }\DecValTok{44650}\NormalTok{, }\DecValTok{45015}\NormalTok{, }\DecValTok{45381}\NormalTok{, }\DecValTok{45746}\NormalTok{, }\DecValTok{46111}\NormalTok{, }\DecValTok{46476}\NormalTok{, }\DecValTok{46842}\NormalTok{)}\CommentTok{\# mid point of year}
\NormalTok{  )}
\end{Highlighting}
\end{Shaded}

\hypertarget{vertical-depth-z-attributes}{%
\subsubsection{Vertical (depth / Z) Attributes}\label{vertical-depth-z-attributes}}

A named list describing the vertical (depth / Z) dimension.

\begin{Shaded}
\begin{Highlighting}[]
\NormalTok{vertical\_attributes \textless{}{-}}\StringTok{ }\KeywordTok{list}\NormalTok{(}
  \DataTypeTok{name =} \StringTok{\textquotesingle{}depth\textquotesingle{}}\NormalTok{,}
  \DataTypeTok{units =} \StringTok{\textquotesingle{}cm\textquotesingle{}}\NormalTok{,}
  \DataTypeTok{positive =} \StringTok{\textquotesingle{}down\textquotesingle{}}\NormalTok{,}
  \DataTypeTok{vals =} \KeywordTok{c}\NormalTok{(}\DecValTok{5}\NormalTok{,}\DecValTok{15}\NormalTok{,}\DecValTok{20}\NormalTok{)}
\NormalTok{)}
\end{Highlighting}
\end{Shaded}

\hypertarget{global-attributes}{%
\subsubsection{Global Attributes}\label{global-attributes}}

A named list desciribing the global attributes

\end{document}
